\documentclass[10pt]{article}
% \usepackage[letterpaper,text={6.5in,8.7in},centering]{geometry}
\usepackage{amssymb,amsmath,times,url,graphicx,amsthm,alltt}
%\usepackage[pdftex,urlcolor=blue,pdfpagemode=none,pdfstartview=FitH]{hyperref}
\usepackage{my_packages}
\usepackage{tikz_packages}
%% url smaller font.
\makeatletter
\def\url@leostyle{%
  \@ifundefined{selectfont}{\def\UrlFont{\sf}}{\def\UrlFont{\small\ttfamily}}}
\makeatother
\urlstyle{leo}

%\usepackage[all,import]{xy}

\renewcommand{\baselinestretch}{1.2}
\date{}

\renewcommand{\thesubsection}{\arabic{subsection}. }
\renewcommand{\thesubsubsection}{\arabic{subsection}.\arabic{subsubsection} }

\theoremstyle{definition}
\newtheorem{prob}{Problem}[section]
%\renewcommand{\theprob}{\arabic{section}.\arabic{prob}}
\renewcommand{\theprob}{\arabic{prob}}

\newenvironment{subprob}%
{\renewcommand{\theenumi}{\alph{enumi}}\renewcommand{\labelenumi}{(\theenumi)}\begin{enumerate}}%
{\end{enumerate}}%


\begin{document}

\pagestyle{empty}
\section*{MAE3134: Homework 0 - Skills Review}
\vspace*{-0.4cm}
\noindent{Due date: TBD}%\\%\vspace*{0.5cm}

\begin{prob}
    Consider the general n-th order ordinary differential equation
    \begin{align*}
        F(t, y(t), y'(t), \ldots, y^{(n)}(t)) = 0.
    \end{align*}

    \begin{subprob}
        \item What general form must \( F \) have for the equation to be linear?

        Classify the following equations as linear or non-linear, state their order, and identify the dependent and independent variables as well as any non-linear terms:
    \item \( t \ddiff{y}{t} + t^2 \diff{y}{t} + t^3 y = \cos t \)
    \item \( t \dddiff{y}{t} + t^2 \diff{y}{t} + t^3 y = \cos y \)
    \item \( \diff{y}{x} = \frac{2 y - 3}{2 x + 2} \) 
    \item \( \parenth{\cos t} \ddiff{y}{t} + \parenth{\sin 2 t} y = 0 \quad y = y(t)\)

        Classify the following equations as \textbf{ordinary} or \textbf{partial} differential equations, also indicate the dependent and independent variables:

    \item \( \diff{x}{t} + \diff{y}{t} + x + y = 0 \quad x=x(t) \quad y = y(t)\)
    \item \( \diff{f}{x} + \diff{f}{y} + x + y = 0 \quad f = f(x, y)\)
    \item \( \diff{}{t} \bracket{\diff{f}{x}} = 0 \quad f = x^2 + \diff{x}{t}\)
    \item \( \diff{f}{x} = x \quad f = y^2(x)  + \diff{y}{x}\)

        Classify the following linear differential equations as either \textbf{time-invariant} or \textbf{time-variable}. Indicate any time-variable terms.
    \item \( \ddiff{y}{t} + 2 y = 0\)
    \item \( \diff{t^2 y}{t} = 0 \)
    \item \( \parenth{\frac{1}{t+1}}\ddiff{y}{t} + \parenth{\frac{1}{t+1}} y = 0 \)
    \item \( \ddiff{y}{t} + \parenth{\cos{t}} y = 0 \)
    \end{subprob}

\end{prob}

\clearpage\newpage

\begin{prob}
    
    \begin{figure}[h]
        \centering
        \begin{scaletikzpicturetowidth}{0.75\textwidth}
            \begin{tikzpicture}[scale=\tikzscale]
                % Define all coordinates. Not strictly necessary, but it
                % makes for cleaner code, in my humble opinion.
                \coordinate[]   (bottom_left) at (0,0);
                \coordinate[]   (top_left) at (0,1);
                \coordinate[]   (A1) at (1,1);
                \coordinate[]   (B1) at (1,0);
                \coordinate[label=below:$+$] (OUTP) at (2, 1);
                \coordinate[label=above:$-$] (OUTN) at (2, 0);

                % now do the drawing
                \draw
                    (top_left) to[american voltage source, l_=\( v_{in} (t) \), *-*] (bottom_left)
                    ;

                \draw (top_left) to[resistor, l=40<\kilo\ohm>, *-*] (A1)
                    to[short, *-*] (OUTP);

                \draw (bottom_left) to[short, *-*] (B1);

                \draw (A1) to[capacitor, l_=5<\micro\farad>] (B1);

                \draw (B1) to[short, *-*] (OUTN);

                \node at ($0.5*(OUTP) + 0.5*(OUTN)$) {$v_{out}(t)$};
    \end{tikzpicture}
    \end{scaletikzpicturetowidth}
        \caption{Electrical System ~\label{fig:electrical_system}}
    \end{figure}
    
    The differential equation relating \( v_{out}(t) \) to \( v_{in}(t) \) for this circuit is given by:
    \begin{align*}
        \diff{v_{out}}{t} + 5 v_{out} = 5 v_{in}(t).
    \end{align*}

    \begin{subprob}
    \item If \( v_{in}(t) = \SI{2}{\volt} \) and \( v_{out}(0) = \SI{0}{\volt} \), find \( v_{out}(t) \) using either the method of undetermined coefficients or the Laplace transform (show your work).
        \vspace*{6cm} \\

    If \( v_{out}(t) = 2 \parenth{ 1 -e^{-5 t} }\) :

    \item What is the steady-state value ( value at \( t \to \infty\)) of \( v_{out}\)?

        \vspace*{2cm}
    \item When does \( v_{out} \) reach \SI{10}{\percent} of its steady-state value?
        \vspace*{2cm}
    \item When does \( v_{out} \) reach \SI{90}{\percent} of its steady-state value?
        \vspace*{2cm}
    \item When does \( v_{out} \) reach \SI{98}{\percent} of its steady-state value?
    \end{subprob}
\end{prob}

\clearpage\newpage
\begin{prob}
    The motion of a particle is described by :
    \begin{align*}
        y = 0.7 \cos \parenth{\frac{\pi}{3}t + \frac{\pi}{6}}
    \end{align*}
    where \( y \) represents the position of the particle in meters and \( t \) is in seconds.
    
    \begin{subprob}
    \item What is the value of the initial displacment?
    \item What is the value of the initial velocity?
    \item What is the initial acceleration?
    \item What is the maximum velocity?
    \item What is the value of \( t \) when \( y \) reaches the first maximum (the first positive peak)?
    \item Using the programming language of your choice (i.e. Matlab, Python etc), generate a plot of the motion for \( t \in [0, 20] \si{\second}\).
    \end{subprob}
\end{prob}

\clearpage\newpage

% \begin{prob}
%     A block with mass \SI{500}{\gram} oscillates at the end of a linear spring with a spring constant of \SI{7.5}{\newton\per\meter}.
%     You can assume this is a top-down view and the block is sliding on a frictionless surface, i.e. gravity is not a factor in this problem.


%     \begin{figure}[h]
%         \centering
%         \begin{scaletikzpicturetowidth}{0.5\textwidth}
%             \begin{tikzpicture}[scale=\tikzscale]
%                 \coordinate[] (origin) at (0, 0);
%                 \coordinate[] (mass) at (1, 0);
                
%                 \node[block,minimum width=2cm, minimum height=2cm] (m) at (mass) {$m$};
%                 \node[ground, rotate=-90, minimum width=2cm] (wall)  at (origin) {};

%                 \draw (wall.north east) -- (wall.north west);
                
%                 % create nodes for the spring to attach to 
%                 \node [virtual, left=of m.180] (sr) {};
%                 \node [virtual, right=of wall.0] (sl) {};

%                 \draw[spring] (origin.east) -- (m.west);
%         \end{tikzpicture}
%         \end{scaletikzpicturetowidth}
%         \caption{Mechanical System~\label{fig:mechanical_system}}
%     \end{figure}

%     \begin{subprob}
%     \item What is the period of the block's motion in seconds?
%     \item If the block's maximum acceleration is \SI{3.0}{\meter\per\second\squared}, what is the amplitude of the motion in \si{\meter}?
%     \end{subprob}
% \end{prob}

% \clearpage\newpage

\begin{prob}
    A linear system's time response is given by
    \begin{align*}
        x(t) = 0.1 e^{-5t} .
    \end{align*}

    By hand, draw an accurate approximation for the plot of \( x(t) \) versus \( t \).
    Label your axes.
\end{prob}

\clearpage\newpage

\begin{prob}
    Given the following matrices:
        \begin{align*}
            A = \begin{bmatrix} 0 & 1 \\ -10 & 2\end{bmatrix} \qquad 
            \bracket{sI-A} = \begin{bmatrix} s & -1 \\ 10 & s- 2 \end{bmatrix}
        \end{align*}

    Determine the following.
    \begin{subprob}
    \item Find the eigenvalues of \( A\).
    \item Find the inverse of \( sI -A \) analytically, and validate your answer.
    \end{subprob}
\end{prob}

\clearpage\newpage

\begin{prob}
    Given the complex numbers \( a = -2 + 0.5 j \) and \( \lambda = -1 + 3 j\).

    \begin{subprob}
    \item What are the complex conjugates of \( a, \lambda\), i.e \( a^*, \lambda^*\)?
    \item Express \( a, a^*\) in polar form.
        Recall: the polar form of a complex number \( a \) is \( \norm{a} e^{j \phi} \) where \( \phi\) is the angle of \( a \) expressed in radians.
    \item Find the complex number \( b = a + \lambda\).
    \item Find the complex number \( c = a \lambda \) (multiplication).
    \item We define the complex plane as the two dimensional plane with the real axis along the horizontal direction and the imaginary axis along the vertical direction.
        For the four complex numbers (\( a, b, c, \lambda\) computed above, plot their location on the complex plane. 
        In addition, mark the angle and radius of each vector on your plot.
    \end{subprob}
\end{prob}
\end{document}

