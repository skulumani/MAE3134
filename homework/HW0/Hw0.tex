\documentclass[10pt]{article}
% \usepackage[letterpaper,text={6.5in,8.7in},centering]{geometry}
\usepackage{amssymb,amsmath,times,url,graphicx,amsthm,alltt}
%\usepackage[pdftex,urlcolor=blue,pdfpagemode=none,pdfstartview=FitH]{hyperref}
\usepackage{my_packages}
\usepackage{tikz_packages}
%% url smaller font.
\makeatletter
\def\url@leostyle{%
  \@ifundefined{selectfont}{\def\UrlFont{\sf}}{\def\UrlFont{\small\ttfamily}}}
\makeatother
\urlstyle{leo}

%\usepackage[all,import]{xy}

\renewcommand{\baselinestretch}{1.2}
\date{}

\renewcommand{\thesubsection}{\arabic{subsection}. }
\renewcommand{\thesubsubsection}{\arabic{subsection}.\arabic{subsubsection} }

\theoremstyle{definition}
\newtheorem{prob}{Problem}[section]
%\renewcommand{\theprob}{\arabic{section}.\arabic{prob}}
\renewcommand{\theprob}{\arabic{prob}}

\newenvironment{subprob}%
{\renewcommand{\theenumi}{\alph{enumi}}\renewcommand{\labelenumi}{(\theenumi)}\begin{enumerate}}%
{\end{enumerate}}%


\begin{document}

\pagestyle{empty}
\section*{MAE3134: Homework 0 - Skills Review}
\vspace*{-0.4cm}
\noindent{Due date: TBD}%\\%\vspace*{0.5cm}

\begin{prob}
    
    \begin{figure}[h]
        \centering
        \begin{scaletikzpicturetowidth}{1\textwidth}
            \begin{tikzpicture}[scale=\tikzscale]
                % Define all coordinates. Not strictly necessary, but it
                % makes for cleaner code, in my humble opinion.
                \coordinate[]   (bottom_left) at (0,0);
                \coordinate[]   (top_left) at (0,1);
                \coordinate[]   (A1) at (1,1);
                \coordinate[]   (B1) at (1,0);
                \coordinate[label=below:$+$] (OUTP) at (2, 1);
                \coordinate[label=above:$-$] (OUTN) at (2, 0);

                % now do the drawing
                \draw
                    (top_left) to[american voltage source, l_=\( v_{in} (t) \), *-*] (bottom_left)
                    ;

                \draw (top_left) to[resistor, l=40<\kilo\ohm>, *-*] (A1)
                    to[short, *-*] (OUTP);


    \end{tikzpicture}
    \end{scaletikzpicturetowidth}
        \caption{Electrical System ~\label{fig:electrical_system}}
    \end{figure}

    The magnitude of a vector is defined as 
    \begin{align*}
        \norm{\vec a} = \sqrt{a_1^2 + a_2^2 + a_3^2} ,
    \end{align*}
    for the vector 
    \begin{align*}
        \vec a = \begin{bmatrix} a_1 & a_2 & a_3 . \end{bmatrix}
    \end{align*}

    \begin{subprob}
        \item Find the magnitude of \( \norm{\vec a} \) where \( \vec{a} = \begin{bmatrix} \sqrt{5} & \sqrt{3} & 1 \end{bmatrix} \).
        \item Find the magnitude of \( \norm{\vec b} \) where \( \vec{b} = \begin{bmatrix} -2 & 4 & -4 \end{bmatrix} \). 
        \item Find the magnitude of \( \norm{\vec c} \) where \( \vec{c} = \begin{bmatrix} 0 & 0 & -9\end{bmatrix} \). 
    \end{subprob}
\end{prob}

\begin{prob}
    Consider two vectors defined as
    \begin{align*}
    \vec a &= \begin{bmatrix} 0 & 3000 & 0 \end{bmatrix} , \\
        \vec b &= \begin{bmatrix} 4000 & 0 & 0 \end{bmatrix}.
    \end{align*}

    \begin{subprob}
    \item Find \( \vec c = \vec a + \vec b \).
    \item Find \( \norm{\vec c} \).
    \item Find \( \norm{\vec a} + \norm{\vec b} \).
    \item True or False. \( \norm{\vec a} + \norm{\vec b} > \norm{\vec c} \).
    \end{subprob}
\end{prob}

\end{document}

