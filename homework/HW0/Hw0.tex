\documentclass[10pt]{article}
% \usepackage[letterpaper,text={6.5in,8.7in},centering]{geometry}
\usepackage{amssymb,amsmath,times,url,graphicx,amsthm,alltt}
%\usepackage[pdftex,urlcolor=blue,pdfpagemode=none,pdfstartview=FitH]{hyperref}
\usepackage{my_packages}
\usepackage{tikz_packages}
%% url smaller font.
\makeatletter
\def\url@leostyle{%
  \@ifundefined{selectfont}{\def\UrlFont{\sf}}{\def\UrlFont{\small\ttfamily}}}
\makeatother
\urlstyle{leo}

%\usepackage[all,import]{xy}

\renewcommand{\baselinestretch}{1.2}
\date{}

\renewcommand{\thesubsection}{\arabic{subsection}. }
\renewcommand{\thesubsubsection}{\arabic{subsection}.\arabic{subsubsection} }

\theoremstyle{definition}
\newtheorem{prob}{Problem}[section]
%\renewcommand{\theprob}{\arabic{section}.\arabic{prob}}
\renewcommand{\theprob}{\arabic{prob}}

\newenvironment{subprob}%
{\renewcommand{\theenumi}{\alph{enumi}}\renewcommand{\labelenumi}{(\theenumi)}\begin{enumerate}}%
{\end{enumerate}}%


\begin{document}

\pagestyle{empty}
\section*{MAE3134: Homework 0 - Skills Review}
\vspace*{-0.4cm}
\noindent{Due date: TBD}%\\%\vspace*{0.5cm}

\begin{prob}
    
    \begin{figure}[h]
        \centering
        \begin{scaletikzpicturetowidth}{1\textwidth}
            \begin{tikzpicture}[scale=\tikzscale]
        % Define all coordinates. Not strictly necessary, but it
        % makes for cleaner code, in my humble opinion.
        \coordinate[label=below:$+$]   (INP) at (0,0);
        \coordinate[label=$A_1$]       (A1)  at (1,0);
        \coordinate[label=$B_1$]       (B1)  at (2,0);
        \coordinate                    (C1)  at (1,-1);
        \coordinate                    (CE1) at (2,-1);
        \coordinate[label=above:$-$]   (INN) at (0,-1);
        \coordinate[label=$D_1$]       (D1)  at (3,0);
        \coordinate                    (L1)  at (3,-1);

        % Draw part of the circuit. You might need more than one
        % draw command, depending on how you do things.
        \draw 
            (INP) to[short,o-*,i_=$i_{ul}$]                (A1)
                to[R,l_=$R_1$,-*,i^>=$i_2$]              (B1)
                to[C,l_=$C_{E1}$,-*,i^>=$i_{e1}$]        (CE1)
                --                                       (C1)
                to[C,l_=$C_1$,*-*,i<^=$i_1$]             (A1)
            (INN) to[short,o-]                             (C1)
            (B1)  to[I,i_<=$gi_1$,color=orange]            (D1)
                to[L,l_=$L_1$,color=magenta,i=$i_4$,*-*] (L1)
                --                                       (CE1)
        ;

        % Some additional labelling...
        \node at (0,-0.5) {$u_{ul}$};

        % Helper lines
        % Remove the 'dashed' parameter for a normal line.
        % This part uses the 'calc' library from TikZ for
        % coordinate calculations.
        % NOTE: The corner radius has to be adjusted manually
        %       if you adjust the base x and y lengths in the
        %       optional argument for the tikzpicture/circuitikz
        %       environment.
        \draw[red,dashed,rounded corners=0.2cm,-latex]
            ($(INP) + ( 0.175,-0.5  )$) 
            -- ($(INP) + ( 0.175,-0.175)$) 
            -- ($(B1)  - ( 0.175, 0.175)$)
            -- ($(CE1) + (-0.175, 0.175)$) 
            -- ($(INN) + ( 0.175, 0.175)$)
        ;
    \end{tikzpicture}
    \end{scaletikzpicturetowidth}
        \caption{Electrical System ~\label{fig:electrical_system}}
    \end{figure}

    The magnitude of a vector is defined as 
    \begin{align*}
        \norm{\vec a} = \sqrt{a_1^2 + a_2^2 + a_3^2} ,
    \end{align*}
    for the vector 
    \begin{align*}
        \vec a = \begin{bmatrix} a_1 & a_2 & a_3 . \end{bmatrix}
    \end{align*}

    \begin{subprob}
        \item Find the magnitude of \( \norm{\vec a} \) where \( \vec{a} = \begin{bmatrix} \sqrt{5} & \sqrt{3} & 1 \end{bmatrix} \).
        \item Find the magnitude of \( \norm{\vec b} \) where \( \vec{b} = \begin{bmatrix} -2 & 4 & -4 \end{bmatrix} \). 
        \item Find the magnitude of \( \norm{\vec c} \) where \( \vec{c} = \begin{bmatrix} 0 & 0 & -9\end{bmatrix} \). 
    \end{subprob}
\end{prob}

\begin{prob}
    Consider two vectors defined as
    \begin{align*}
    \vec a &= \begin{bmatrix} 0 & 3000 & 0 \end{bmatrix} , \\
        \vec b &= \begin{bmatrix} 4000 & 0 & 0 \end{bmatrix}.
    \end{align*}

    \begin{subprob}
    \item Find \( \vec c = \vec a + \vec b \).
    \item Find \( \norm{\vec c} \).
    \item Find \( \norm{\vec a} + \norm{\vec b} \).
    \item True or False. \( \norm{\vec a} + \norm{\vec b} > \norm{\vec c} \).
    \end{subprob}
\end{prob}

\end{document}

