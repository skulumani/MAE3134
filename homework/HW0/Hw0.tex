\documentclass[10pt]{article}
% \usepackage[letterpaper,text={6.5in,8.7in},centering]{geometry}
\usepackage{amssymb,amsmath,times,url,graphicx,amsthm,alltt}
%\usepackage[pdftex,urlcolor=blue,pdfpagemode=none,pdfstartview=FitH]{hyperref}
\usepackage{my_packages}
\usepackage{tikz_packages}
%% url smaller font.
\makeatletter
\def\url@leostyle{%
  \@ifundefined{selectfont}{\def\UrlFont{\sf}}{\def\UrlFont{\small\ttfamily}}}
\makeatother
\urlstyle{leo}

%\usepackage[all,import]{xy}

\renewcommand{\baselinestretch}{1.2}
\date{}

\renewcommand{\thesubsection}{\arabic{subsection}. }
\renewcommand{\thesubsubsection}{\arabic{subsection}.\arabic{subsubsection} }

\theoremstyle{definition}
\newtheorem{prob}{Problem}[section]
%\renewcommand{\theprob}{\arabic{section}.\arabic{prob}}
\renewcommand{\theprob}{\arabic{prob}}

\newenvironment{subprob}%
{\renewcommand{\theenumi}{\alph{enumi}}\renewcommand{\labelenumi}{(\theenumi)}\begin{enumerate}}%
{\end{enumerate}}%


\begin{document}

\pagestyle{empty}
\section*{MAE3134: Homework 0 - Skills Review}
\vspace*{-0.4cm}
\noindent{Due date: TBD}%\\%\vspace*{0.5cm}

\begin{prob}
    
    \begin{figure}[h]
        \centering
        \begin{scaletikzpicturetowidth}{0.75\textwidth}
            \begin{tikzpicture}[scale=\tikzscale]
                % Define all coordinates. Not strictly necessary, but it
                % makes for cleaner code, in my humble opinion.
                \coordinate[]   (bottom_left) at (0,0);
                \coordinate[]   (top_left) at (0,1);
                \coordinate[]   (A1) at (1,1);
                \coordinate[]   (B1) at (1,0);
                \coordinate[label=below:$+$] (OUTP) at (2, 1);
                \coordinate[label=above:$-$] (OUTN) at (2, 0);

                % now do the drawing
                \draw
                    (top_left) to[american voltage source, l_=\( v_{in} (t) \), *-*] (bottom_left)
                    ;

                \draw (top_left) to[resistor, l=40<\kilo\ohm>, *-*] (A1)
                    to[short, *-*] (OUTP);

                \draw (bottom_left) to[short, *-*] (B1);

                \draw (A1) to[capacitor, l_=5<\micro\farad>] (B1);

                \draw (B1) to[short, *-*] (OUTN);

                \node at ($0.5*(OUTP) + 0.5*(OUTN)$) {$v_{out}(t)$};
    \end{tikzpicture}
    \end{scaletikzpicturetowidth}
        \caption{Electrical System ~\label{fig:electrical_system}}
    \end{figure}
    
    The differential equation relating \( v_{out}(t) \) to \( v_{in}(t) \) for this circuit is given by:
    \begin{align*}
        \diff{v_{out}}{t} + 5 v_{out} = 5 v_{in}(t).
    \end{align*}

    \begin{subprob}
    \item If \( v_{in}(t) = \SI{2}{\volt} \) and \( v_{out}(0) = \SI{0}{\volt} \), find \( v_{out}(t) \) using either the method of undetermined coefficients of the Laplace transform (show your work).
        \vspace*{6cm} \\

    If \( v_{out}(t) = 2 \parenth{ 1 -e^{-5 t} }\) :

    \item What is the steady-state value ( value at \( t \to \infty\)) of \( v_{out}\)?

        \vspace*{2cm}
    \item When does \( v_{out} \) reach \SI{10}{\percent} of its steady-state value?
        \vspace*{2cm}
    \item When does \( v_{out} \) reach \SI{90}{\percent} of its steady-state value?
        \vspace*{2cm}
    \item When does \( v_{out} \) reach \SI{98}{\percent} of its steady-state value?
    \end{subprob}
\end{prob}


\end{document}

