
\documentclass[10pt]{article}
% \usepackage[letterpaper,text={6.5in,8.7in},centering]{geometry}
\usepackage{amssymb,amsmath,times,url,graphicx,amsthm,alltt}
%\usepackage[pdftex,urlcolor=blue,pdfpagemode=none,pdfstartview=FitH]{hyperref}
\usepackage{my_packages}
\usepackage{tikz_packages}
%% url smaller font.
\makeatletter
\def\url@leostyle{%
  \@ifundefined{selectfont}{\def\UrlFont{\sf}}{\def\UrlFont{\small\ttfamily}}}
\makeatother
\urlstyle{leo}

%\usepackage[all,import]{xy}

\renewcommand{\baselinestretch}{1.2}
\date{}

\renewcommand{\thesubsection}{\arabic{subsection}. }
\renewcommand{\thesubsubsection}{\arabic{subsection}.\arabic{subsubsection} }

\theoremstyle{definition}
\newtheorem{prob}{Problem}[section]
%\renewcommand{\theprob}{\arabic{section}.\arabic{prob}}
\renewcommand{\theprob}{\arabic{prob}}

\newenvironment{subprob}%
{\renewcommand{\theenumi}{\alph{enumi}}\renewcommand{\labelenumi}{(\theenumi)}\begin{enumerate}}%
{\end{enumerate}}%


\begin{document}

\pagestyle{empty}
\section*{MAE3134: Homework 4}
\vspace*{-0.4cm}
\noindent{Due date: 6 March 2018}%\\%\vspace*{0.5cm}

\begin{prob}
    For each second order system below accomplish the following:
    \begin{enumerate}
        \item Find \( \zeta , \omega_n, T_s, T_p, T_r, \text{ and } \% OS\).
        \item Use Python/Matlab to generate a plot of the response and mark the response specifications on your plots.
    \end{enumerate}
    You will need to estimate the rise time from a plot of the unit step response.
    \begin{subprob}
        \item \[ T(s) = \frac{16}{s^2 + 3s + 16}\]
        \item \[ T(s) = \frac{0.04}{s^2 + 0.02 s + 0.04}\]
        \item \[ T(s) = \frac{\num{1.05e7}}{s^2 + \num{1.6e3}s + \num{1.05e7}}\]
    \end{subprob}
\end{prob}

\begin{prob}
    For each pair of second order system specifications accomplish the following:
    \begin{enumerate}
        \item Find the regions on the complex plane that will meet the system specifications, i.e. draw the complex plane and identify the locations that will meet the requirements.
        \item Write the transfer function that will satisfy the specifications.
        \item Verify that your system will meet the specificiations by generating the response  to a unit step input.
        \item On the plots mark the various response specifications.
    \end{enumerate}
    \begin{subprob}
        \item \( \% OS = 12 \%, T_s = \SI{0.6}{\second}\)
        \item \( \% OS = 10 \%, T_p = \SI{5}{\second}\)
        \item \( T_s = \SI{7}{\second}, T_p = \SI{3}{\second}\)
    \end{subprob}
\end{prob}


\begin{prob}
   Find the  state space represenation for the following transfer functions.
   \begin{subprob}
        \item  
            \begin{align*}
                \frac{C(s)}{R(s)} = \frac{24}{s^3 + 9s^2 + 26s + 24}
            \end{align*}
        \item 
            \begin{align*}
                \frac{C(s)}{R(s)} = \frac{s^2 + 7s + 2}{s^3 + 9s^2 + 26s + 24}
            \end{align*}

        \item 
            \begin{align*}
                G(s) = \frac{2s + 1}{s^2 + 7s +9}
            \end{align*}
   \end{subprob}
\end{prob}

\clearpage\newpage
\begin{prob}
    Find the transfer function for the following systems.
    Assume the output is \( Y(s) \) and the input is \( U(s) \).

    \begin{subprob}
        \item 
            \begin{align*}
                \dot{\vecbf{x}} &= 
                \begin{bmatrix} 0 & 1 & 0 \\
                    0 & 0 & 1 \\
                    -1 & -2 & -3
                \end{bmatrix} \vecbf{x} +
                \begin{bmatrix} 10 \\ 0 \\ 0\end{bmatrix} u \\
                y &= \begin{bmatrix} 1 & 0 & 0 \end{bmatrix} \vecbf{x}
            \end{align*}
        \item 
            \begin{align*}
                \dot{\vecbf{x}} &= 
                \begin{bmatrix}-4 & -1.5 \\ 4 & 0\end{bmatrix} \vecbf{x} +
                \begin{bmatrix}2 \\ 0\end{bmatrix} u \\
                y &= \begin{bmatrix}1.5 & 0.625\end{bmatrix} \vecbf{x}
            \end{align*}
    \end{subprob}
\end{prob}
\end{document}
