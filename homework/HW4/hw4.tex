
\documentclass[10pt]{article}
% \usepackage[letterpaper,text={6.5in,8.7in},centering]{geometry}
\usepackage{amssymb,amsmath,times,url,graphicx,amsthm,alltt}
%\usepackage[pdftex,urlcolor=blue,pdfpagemode=none,pdfstartview=FitH]{hyperref}
\usepackage{my_packages}
\usepackage{tikz_packages}
%% url smaller font.
\makeatletter
\def\url@leostyle{%
  \@ifundefined{selectfont}{\def\UrlFont{\sf}}{\def\UrlFont{\small\ttfamily}}}
\makeatother
\urlstyle{leo}

%\usepackage[all,import]{xy}

\renewcommand{\baselinestretch}{1.2}
\date{}

\renewcommand{\thesubsection}{\arabic{subsection}. }
\renewcommand{\thesubsubsection}{\arabic{subsection}.\arabic{subsubsection} }

\theoremstyle{definition}
\newtheorem{prob}{Problem}[section]
%\renewcommand{\theprob}{\arabic{section}.\arabic{prob}}
\renewcommand{\theprob}{\arabic{prob}}

\newenvironment{subprob}%
{\renewcommand{\theenumi}{\alph{enumi}}\renewcommand{\labelenumi}{(\theenumi)}\begin{enumerate}}%
{\end{enumerate}}%


\begin{document}

\pagestyle{empty}
\section*{MAE3134: Homework 3}
\vspace*{-0.4cm}
\noindent{Due date: 20 February 2018}%\\%\vspace*{0.5cm}
\begin{prob}
    A Boeing 747 is in Straight and Level Unaccelerated Flight (SLUF) at \( \SI{40000}{ft}, \SI{871}{ft \per \second}\).
    The linearized equations of longitudanal motion are given by
    \begin{align}
        \begin{bmatrix}
            \begin{array}{ccc}
                s + 0.0828 & -0.0215 & 0.5589 \\
                0.0573 & 15.3348s + 5.9633 & -15.0685s \\
                0.0057 & 0.1425s + 1.6165 & s^2 + 0.0438s
            \end{array}
        \end{bmatrix}
        \begin{bmatrix}
            \begin{array}{c}
                \delta u \\ \delta \alpha \\ \delta \theta
            \end{array}
        \end{bmatrix}
        =
        \begin{bmatrix}
            \begin{array}{c}
                0 \\ -0.3226 \\ -1.2124
            \end{array}
        \end{bmatrix}
        \delta e ,
    \end{align}
    where \( \delta u, \delta \theta, \delta e, \delta \alpha\) are the deviations of forward velocity in \si{ft \per \second}, pitch in degrees, elevation deflection in degree, and deviation of the angle of attack in degrees, respectively.

We can define a transfer function relating the input elevator deflection to the output of the pitch devaitions as
\begin{align}
    \frac{\theta(s)}{\delta e(s)} = \frac{-18.5 (s + 0.36) (s+0.08)}{(s + 0.47 \pm 1.24 j)(s+0.06)(s+0.02)} .
\end{align}

\begin{subprob}
    \item Find the pitch response \( \theta (t) \) if the elevation is deflected a constant \SI{-0.01}{\degree}.
\end{subprob}
\end{prob}

It is typically difficult to analyze the short period motion, due to the complex poles, and the phugoid motion, due to the real poles. 
As a result, it is possible to seperate the dynamics into two seperate transfer functions:
\begin{align}
    \frac{\delta \alpha}{\delta e} &= \frac{-0.021 (s^2 + 56.7 s)}{s(s+0.283 \pm 1.24 j)} \\
    \frac{\delta \theta}{\delta e} &= \frac{-1.21 (s + 0.36)}{s(s+0.283 \pm 1.24 j)}
\end{align}

\begin{prob}
    Insert a tikz picture of a missle in flight
    A missle in flight, as shown in Fig whatever, is subject to several forces, such as thrust, lift, drag, and gravity. 
    The missle flies at an angle of attack, \( \alpha \), with respect to the velocity vector, creating lift.
    For steering, the body angle from vertical , \( \phi\), is controlled by rotating the thrust vector at the tail.
    The transfer function relating the body angle, \( \phi\), to the angular displacement, \( \delta \), of the engine is of the form
    \begin{align}
        \frac{\phi(s)}{\delta (s)} = \frac{K_a s + K_b}{K_3 s^3 + K_s s^2 + K_1 s + K_0}
    \end{align}
\end{prob}

\begin{prob}
    For each second order system below, find \( \zeta , \omega_n, T_s, T_p, T_r, and \% OS\).
    You will need to estimate the rise time from a plot of the unit step response.
    \begin{subprob}
        \item \[ T(s) = \frac{16}{s^2 + 3s + 16}\]
        \item \[ T(s) = \frac{0.04}{s^2 + 0.02 s + 0.04}\]
        \item \[ T(s) = \frac{\num{1.05e7}}{s^2 + \num{1.6e3}s + \num{1.05e7}}\]
    \end{subprob}
\end{prob}

\begin{prob}
    For each pair of second order system specifications, find the location of the second order poles.
    Write the transfer function that will satisfy the specifications.
    \begin{subprob}
        \item \( \% OS = 12 \%, T_s = \SI{0.6}{\second}\)
        \item \( \% OS = 10 \%, T_p = \SI{5}{\second}\)
        \item \( T_s = \SI{7}{\second}, T_p = \SI{3}{\second}\)
    \end{subprob}
\end{prob}
\end{document}
