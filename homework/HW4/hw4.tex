
\documentclass[10pt]{article}
% \usepackage[letterpaper,text={6.5in,8.7in},centering]{geometry}
\usepackage{amssymb,amsmath,times,url,graphicx,amsthm,alltt}
%\usepackage[pdftex,urlcolor=blue,pdfpagemode=none,pdfstartview=FitH]{hyperref}
\usepackage{my_packages}
\usepackage{tikz_packages}
%% url smaller font.
\makeatletter
\def\url@leostyle{%
  \@ifundefined{selectfont}{\def\UrlFont{\sf}}{\def\UrlFont{\small\ttfamily}}}
\makeatother
\urlstyle{leo}

%\usepackage[all,import]{xy}

\renewcommand{\baselinestretch}{1.2}
\date{}

\renewcommand{\thesubsection}{\arabic{subsection}. }
\renewcommand{\thesubsubsection}{\arabic{subsection}.\arabic{subsubsection} }

\theoremstyle{definition}
\newtheorem{prob}{Problem}[section]
%\renewcommand{\theprob}{\arabic{section}.\arabic{prob}}
\renewcommand{\theprob}{\arabic{prob}}

\newenvironment{subprob}%
{\renewcommand{\theenumi}{\alph{enumi}}\renewcommand{\labelenumi}{(\theenumi)}\begin{enumerate}}%
{\end{enumerate}}%


\begin{document}

\pagestyle{empty}
\section*{MAE3134: Homework 3}
\vspace*{-0.4cm}
\noindent{Due date: 20 February 2018}%\\%\vspace*{0.5cm}

\begin{prob}
    For each second order system below, find \( \zeta , \omega_n, T_s, T_p, T_r, and \% OS\).
    You will need to estimate the rise time from a plot of the unit step response.
    \begin{subprob}
        \item \[ T(s) = \frac{16}{s^2 + 3s + 16}\]
        \item \[ T(s) = \frac{0.04}{s^2 + 0.02 s + 0.04}\]
        \item \[ T(s) = \frac{\num{1.05e7}}{s^2 + \num{1.6e3}s + \num{1.05e7}}\]
    \end{subprob}
\end{prob}

\begin{prob}
    For each pair of second order system specifications, find the location of the second order poles.
    Write the transfer function that will satisfy the specifications.
    \begin{subprob}
        \item \( \% OS = 12 \%, T_s = \SI{0.6}{\second}\)
        \item \( \% OS = 10 \%, T_p = \SI{5}{\second}\)
        \item \( T_s = \SI{7}{\second}, T_p = \SI{3}{\second}\)
    \end{subprob}
\end{prob}

\begin{prob}
    Consider the mechanical system shown in~\cref{fig:trans_system}.
    You can assume that the masses roll on frictionless wheels.
    \begin{figure}[h]
        \centering
        \begin{tikzpicture}[every node/.style={outer sep=0pt,thick}, scale=1]

            \node (M1) [draw,minimum width=1cm, minimum height=1cm] {$m_1$};
            \node (M2) [draw,minimum width=1cm, minimum height=1cm] at (2, 0) {$m_2$};

            % ground for M1 and wheels
            \node (ground) [ground,anchor=north,yshift=-0.25cm,minimum width=1.5cm] at (M1.south) {};
            \draw (ground.north east) -- (ground.north west);
            \draw [thick] (M1.south west) ++ (0.2cm,-0.125cm) circle (0.125cm)  (M1.south east) ++ (-0.2cm,-0.125cm) circle (0.125cm);

            % ground for M2 and wheels
            \node (ground) [ground,anchor=north,yshift=-0.25cm,minimum width=1.5cm] at (M2.south) {};
            \draw (ground.north east) -- (ground.north west);
            \draw [thick] (M2.south west) ++ (0.2cm,-0.125cm) circle (0.125cm)  (M2.south east) ++ (-0.2cm,-0.125cm) circle (0.125cm);

            \node[ground, rotate=-90, minimum width=2cm] (wall) at (-2, 0) {};
            \draw (wall.north east) -- (wall.north west);

            % \draw [spring] (wall.145) -- ($(M1.north west)!(wall.145)!(M1.south west)$);
            \draw [damper] (wall.north) -- node[above, yshift=0.25cm] {$D$} (M1.west);
            \draw [spring] (M1.east) -- node[above, yshift=0.25cm] {$K$} (M2.west);

            % arrows for directions and things
            \draw[thick, dashed] ($(M1.north west)$) -- ($(M1.north west) + (0,1)$);
            \draw[thick, dashed] ($(M2.north west)$) -- ($(M2.north west) + (0,1)$);

            \draw[ultra thick, -latex] ($(M2.north west) + (0,0.75)$) -- 
                ($(M2.north west) + (1,0.75)$)
                node [midway, below] {$x_2$};
            \draw[ultra thick, -latex] ($(M1.north west) + (0,0.75)$) -- 
                ($(M1.north west) + (1,0.75)$)
                node [midway, below] {$x_1$};
            \draw[ultra thick, -latex] (M2.east) -- ($(M2.east) + (1, 0)$) 
                node [right] {$f(t)$}; 
        \end{tikzpicture}
        \caption{Translational Mechanical System~\label{fig:trans_system}}
    \end{figure}

    \begin{subprob}
        \item Find the differential equations of motion for this system.
        \item Defining the state of the system as
            \begin{align*}
                \vecbf{x} = \begin{bmatrix} x_1 & \dot{x}_1 & x_2 & \dot{x}_2 \end{bmatrix}^T ,
            \end{align*}
            find the state equation for this mechanical system.
        \item Find the output equation if the desired output is \( x_2(t) \).
    \end{subprob}
\end{prob}

\clearpage\newpage
\begin{prob}
    Consider the mechanical system in~\cref{fig:triple_mass}.
    Assume that all components have a numerical  value of \( 1 \).
    
    \begin{figure}[h]
        \centering
\begin{tikzpicture}[every node/.style={outer sep=0pt,thick}]

\node (M1) [draw,minimum width=1cm, minimum height=1cm] {$m_1$};
\node (M2) [draw,minimum width=1cm, minimum height=1cm] at (2, 0) {$m_2$};
\node (M3) [draw,minimum width=1cm, minimum height=1cm] at (4, 0) {$m_3$};

% ground for M1 and wheels
\node (ground) [ground,anchor=north,yshift=-0.25cm,minimum width=1.5cm] at (M1.south) {};
\draw (ground.north east) -- (ground.north west);
\draw [thick] (M1.south west) ++ (0.2cm,-0.125cm) circle (0.125cm)  (M1.south east) ++ (-0.2cm,-0.125cm) circle (0.125cm);

% ground for M2 and wheels
\node (ground) [ground,anchor=north,yshift=-0.25cm,minimum width=1.5cm] at (M2.south) {};
\draw (ground.north east) -- (ground.north west);
\draw [thick] (M2.south west) ++ (0.2cm,-0.125cm) circle (0.125cm)  (M2.south east) ++ (-0.2cm,-0.125cm) circle (0.125cm);

% ground for M3 and wheels
\node (ground) [ground,anchor=north,yshift=-0.25cm,minimum width=1.5cm] at (M3.south) {};
\draw (ground.north east) -- (ground.north west);
\draw [thick] (M3.south west) ++ (0.2cm,-0.125cm) circle (0.125cm)  (M3.south east) ++ (-0.2cm,-0.125cm) circle (0.125cm);

\node[ground, rotate=-90, minimum width=2cm] (wall) at (-2, 0) {};
\draw (wall.north east) -- (wall.north west);

\node[ground, rotate=90, minimum width=2cm] (rwall) at (6, 0) {};
\draw (rwall.north east) -- (rwall.north west);

% \draw [spring] (wall.145) -- ($(M1.north west)!(wall.145)!(M1.south west)$);
\draw [spring] (wall.north) -- node[above, yshift=0.25cm] {$k_1$} (M1.west);
\draw [damper] (M1.east) -- node[above, yshift=0.25cm] {$d_1$} (M2.west);
\draw [spring] (M2.east) -- node[above, yshift=0.25cm] {$k_2$} (M3.west);
\draw [damper] (M3.east) -- node[above, yshift=0.25cm] {$d_2$} (rwall.north);

% arrows for directions and things
\draw[thick, dashed] ($(M1.north west)$) -- ($(M1.north west) + (0,1)$);
\draw[thick, dashed] ($(M2.north west)$) -- ($(M2.north west) + (0,1)$);
\draw[thick, dashed] ($(M3.north west)$) -- ($(M3.north west) + (0,1)$);

\draw[ultra thick, -latex] ($(M1.north west) + (0,0.75)$) -- 
    ($(M1.north west) + (1,0.75)$)
    node [midway, below] {$x_1$};
\draw[ultra thick, -latex] ($(M2.north west) + (0,0.75)$) -- 
    ($(M2.north west) + (1,0.75)$)
    node [midway, below] {$x_2$};
\draw[ultra thick, -latex] ($(M3.north west) + (0,0.75)$) -- 
    ($(M3.north west) + (1,0.75)$)
    node [midway, below] {$x_3$};

\draw[ultra thick, -latex]  ($(M1.north west) + (-1,0.75)$) node[above left] {$f(t)$} -- ($(M1.north west) + (0, 0.75)$); 
\end{tikzpicture}
    \caption{Triple Mass system~\label{fig:triple_mass}}
    \end{figure}

    \begin{subprob}
        \item Find the differential equation for the system.
        \item Find the state equation assuming the state is defined as
            \begin{align*}
                \vecbf{x} = \begin{bmatrix} x_1 & \dot{x}_1 & x_2 & \dot{x}_2 & x_3 & \dot{x}_3\end{bmatrix}^T
            \end{align*}

        \item Find the output equation if \( x_3(t)\) is the desired output.
    \end{subprob}
\end{prob}

\begin{prob}
   Find the  state space represenation for the following transfer functions.
   \begin{subprob}
        \item  
            \begin{align*}
                \frac{C(s)}{R(s)} = \frac{24}{s^3 + 9s^2 + 26s + 24}
            \end{align*}
        \item 
            \begin{align*}
                \frac{C(s)}{R(s)} = \frac{2^2 + 7s + 2}{s^3 + 9s^2 + 26s + 24}
            \end{align*}

        \item 
            \begin{align*}
                G(s) = \frac{2s + 1}{s^2 + 7s +9}
            \end{align*}
   \end{subprob}
\end{prob}

\begin{prob}
    Find the transfer function for the following systems.
    Assume the output is \( Y(s) \) and the input is \( U(s) \).

    \begin{subprob}
        \item 
            \begin{align*}
                \dot{\vecbf{x}} &= 
                \begin{bmatrix} 0 & 1 & 0 \\
                    0 & 0 & 1 \\
                    -1 & -2 & -3
                \end{bmatrix} \vecbf{x} +
                \begin{bmatrix} 10 \\ 0 \\ 0\end{bmatrix} u \\
                y &= \begin{bmatrix} 1 & 0 & 0 \end{bmatrix} \vecbf{x}
            \end{align*}
        \item 
            \begin{align*}
                \dot{\vecbf{x}} &= 
                \begin{bmatrix}-4 & -1.5 \\ 4 & 0\end{bmatrix} \vecbf{x} +
                \begin{bmatrix}2 \\ 0\end{bmatrix} u \\
                y &= \begin{bmatrix}1.5 & 0.625\end{bmatrix} \vecbf{x}
            \end{align*}
    \end{subprob}
\end{prob}
\end{document}
