\documentclass[10pt]{article}
% \usepackage[letterpaper,text={6.5in,8.7in},centering]{geometry}
\usepackage{amssymb,amsmath,times,url,graphicx,amsthm,alltt}
%\usepackage[pdftex,urlcolor=blue,pdfpagemode=none,pdfstartview=FitH]{hyperref}
\usepackage{my_packages}
\usepackage{tikz_packages}
%% url smaller font.
\makeatletter
\def\url@leostyle{%
  \@ifundefined{selectfont}{\def\UrlFont{\sf}}{\def\UrlFont{\small\ttfamily}}}
\makeatother
\urlstyle{leo}

%\usepackage[all,import]{xy}

\renewcommand{\baselinestretch}{1.2}
\date{}

\renewcommand{\thesubsection}{\arabic{subsection}. }
\renewcommand{\thesubsubsection}{\arabic{subsection}.\arabic{subsubsection} }

\theoremstyle{definition}
\newtheorem{prob}{Problem}[section]
%\renewcommand{\theprob}{\arabic{section}.\arabic{prob}}
\renewcommand{\theprob}{\arabic{prob}}

\newenvironment{subprob}%
{\renewcommand{\theenumi}{\alph{enumi}}\renewcommand{\labelenumi}{(\theenumi)}\begin{enumerate}}%
{\end{enumerate}}%


\begin{document}

\pagestyle{empty}
\section*{MAE3134: Homework 1 - Modelling}
\vspace*{-0.4cm}
\noindent{Due date: TBD}%\\%\vspace*{0.5cm}


\begin{prob}
    A block with mass \SI{500}{\gram} oscillates at the end of a linear spring with a spring constant of \SI{7.5}{\newton\per\meter}.
    You can assume this is a top-down view and the block is sliding on a frictionless surface, i.e. gravity is not a factor in this problem.


    \begin{figure}[h]
        \centering
        \begin{scaletikzpicturetowidth}{0.5\textwidth}
            \begin{tikzpicture}[scale=\tikzscale]
                \coordinate[] (origin) at (0, 0);
                \coordinate[] (mass) at (1, 0);
                
                \node[block,minimum width=2cm, minimum height=2cm] (m) at (mass) {$m$};
                \node[ground, rotate=-90, minimum width=2cm] (wall)  at (origin) {};

                \draw (wall.north east) -- (wall.north west);
                
                % create nodes for the spring to attach to 
                \node [virtual, left=of m.180] (sr) {};
                \node [virtual, right=of wall.0] (sl) {};

                \draw[spring] (origin.east) -- node [below] {$k$} (m.west) ;
        \end{tikzpicture}
        \end{scaletikzpicturetowidth}
        \caption{Mechanical System~\label{fig:mechanical_system}}
    \end{figure}

    \begin{subprob}
    \item What is the equation of motion governing the motion of the block?
    \end{subprob}
\end{prob}


\begin{prob}


    Find the equations of motion for the model given in~\cref{fig:mech2}.
    \begin{figure}[htbp]
        \centering

        \begin{tikzpicture}[every node/.style={outer sep=0pt,thick}]

            \node (M) [draw,minimum width=1cm, minimum height=1cm] {$m$};

            \node (ground) [ground,anchor=north,yshift=-0.25cm,minimum width=1.5cm] at (M.south) {};
            \draw (ground.north east) -- (ground.north west);
            \draw [thick] (M.south west) ++ (0.2cm,-0.125cm) circle (0.125cm)  (M.south east) ++ (-0.2cm,-0.125cm) circle (0.125cm);

            \node[ground, rotate=-90, minimum width=2cm] (wall) at (-2, 0) {};
            \draw (wall.north east) -- (wall.north west);

            \node[ground, rotate=90, minimum width=2cm] (rwall) at (2, 0) {};
            \draw (rwall.north east) -- (rwall.north west);

            \draw [spring] (wall.145) -- node [above, yshift=0.5em] {$k_1$} ($(M.north west)!(wall.145)!(M.south west)$);
            \draw [damper] (wall.30) -- node [below, yshift=-0.5em] {$b_1$} ($(M.north west)!(wall.30)!(M.south west)$);

            \draw [spring] (rwall.30) -- node[above, yshift=0.5em] {$k_2$} ($(M.north east)!(rwall.30)!(M.south east)$);
            \draw [damper] (rwall.145) -- node [below, yshift=-0.5em] {$b_2$} ($(M.north east)!(rwall.145)!(M.south east)$);
            % \draw [-latex,ultra thick] (M.east) ++ (0.2cm,0) -- +(1cm,0);

            \draw[thick, dashed] ($(M.north west)$) -- ($(M.north west) + (0,1)$);
            \draw[ultra thick, -latex] ($(M.north west) + (0,0.75)$) -- 
                ($(M.north west) + (1,0.75)$) node[above] {$x$};
        \end{tikzpicture}

        \caption{Mechanical System\label{fig:mech2}}
    \end{figure}
\end{prob}

\begin{prob}
    George Washington University is attempting to launch its first satellite into orbit, GWSAT.
    In order to control the orientation of the spacecraft, attitude control thrusters are mounted to control the rotational motion about the pitch axis.
    The thrusters are fired in pairs to produce positive and negative torques as required.
    Develop the equations of motion that relates the pitch motion of the spacecraft, \( \theta \), to the thruster force inputs.

\end{prob}

\begin{prob}

Problem 2 from previous semester HW1
\begin{tikzpicture}[every node/.style={thick}]

    \node (M) [draw,minimum width=1cm, minimum height=1cm] {$m$};
    \draw [thick] (M.south west) ++ (0.2cm,-0.125cm) circle (0.125cm)  (M.south east) ++ (-0.2cm,-0.125cm) circle (0.125cm);

    \node (ground) [ground,anchor=north,yshift=-0.25cm,minimum width=1.5cm] at (M.south) {};
    \draw (ground.north east) -- (ground.north west);

    \node[ground, rotate=-90, minimum width=2cm] (wall) at (-2, 0) {};
    \draw (wall.north east) -- (wall.north west);

    \node[ground, rotate=90, minimum width=2cm] (rwall) at (2, 0) {};
    \draw (rwall.north east) -- (rwall.north west);

    \draw [spring] (wall.145) -- node [above, yshift=0.5em] {$k_1$} ($(M.north west)!(wall.145)!(M.south west)$);
    \draw [damper] (wall.30) -- node [below, yshift=-0.5em] {$b_1$} ($(M.north west)!(wall.30)!(M.south west)$);

    \draw [spring] (rwall.30) -- node[above, yshift=0.5em] {$k_2$} ($(M.north east)!(rwall.30)!(M.south east)$);
    \draw [damper] (rwall.145) -- node [below, yshift=-0.5em] {$b_2$} ($(M.north east)!(rwall.145)!(M.south east)$);
    % \draw [-latex,ultra thick] (M.east) ++ (0.2cm,0) -- +(1cm,0);

    \draw[thick, dashed] ($(M.north west)$) -- ($(M.north west) + (0,1)$);
    \draw[ultra thick, -latex] ($(M.north west) + (0,0.75)$) -- 
        ($(M.north west) + (1,0.75)$) node[above] {$x$};
\end{tikzpicture}
\end{prob}
\end{document}

