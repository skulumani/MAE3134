\documentclass[10pt]{article}
% \usepackage[letterpaper,text={6.5in,8.7in},centering]{geometry}
\usepackage{amssymb,amsmath,times,url,graphicx,amsthm,alltt}
%\usepackage[pdftex,urlcolor=blue,pdfpagemode=none,pdfstartview=FitH]{hyperref}
\usepackage{my_packages}
\usepackage{tikz_packages}
%% url smaller font.
\makeatletter
\def\url@leostyle{%
  \@ifundefined{selectfont}{\def\UrlFont{\sf}}{\def\UrlFont{\small\ttfamily}}}
\makeatother
\urlstyle{leo}

%\usepackage[all,import]{xy}

\renewcommand{\baselinestretch}{1.2}
\date{}

\renewcommand{\thesubsection}{\arabic{subsection}. }
\renewcommand{\thesubsubsection}{\arabic{subsection}.\arabic{subsubsection} }

\theoremstyle{definition}
\newtheorem{prob}{Problem}[section]
%\renewcommand{\theprob}{\arabic{section}.\arabic{prob}}
\renewcommand{\theprob}{\arabic{prob}}

\newenvironment{subprob}%
{\renewcommand{\theenumi}{\alph{enumi}}\renewcommand{\labelenumi}{(\theenumi)}\begin{enumerate}}%
{\end{enumerate}}%


\begin{document}

\pagestyle{empty}
\section*{MAE3134: Homework 3}
\vspace*{-0.4cm}
\noindent{Due date: 20 February 2018}%\\%\vspace*{0.5cm}

\begin{prob}
    A low pass filter is modeled using using the following transfer fuction:
    \begin{align}
        \bracket{s + \frac{R_1 + R_2}{R_1 R_2 C}} V_{out}(s) = \frac{1}{R_1 C} V_{in}(s) + V_{out}(0), 
    \end{align}
    where \( R_1 = \SI{40}{\kilo\ohm}, R_2 = \SI{200}{\kilo\ohm}, C = \SI{5}{\micro\farad}\).

    \begin{subprob}
        \item Find the output voltage from the low pass filter assuming an input of \( V_{in}(t) = \SI{2}{\volt}\) and zero initial conditions.
        \item Use the initial and final value theorems to validate your solution.
        \item Generate a plot of the response.
    \end{subprob}
\end{prob}

\begin{prob}
    A single axis accelerometer is modeled using the following transfer function:
    \begin{align}
        \parenth{s^2 + \frac{b}{m_{pm}} + \frac{k}{m_{pm}}} X(s) = \frac{F(s)}{m_c} + s x(0) + \parenth{\dot{x}(0) + \frac{b}{m_{pm}} x(0)},
    \end{align}
    with the following parameters:
    \begin{align*}
        m_c = \SI{1}{\kilo\gram} \quad m_{pm} = \SI{0.01}{\kilo\gram} \quad b = \SI{0.2}{\newton\second\per\meter} \quad k = \SI{1}{\newton\per\meter}
    \end{align*}

    \begin{subprob}
        \item Find the output response \( x(t) \) if \( f(t) = \SI{30}{\newton}\) and the initial conditions are zero.
        \item How does the response change if \( x(0) = \SI{0.1}{\meter} \) and \( \dot{x}(0) = 0 \)?
        \item Use the initial and final value theorems to validate your solutions to the previous questions.
        \item You discover that you'd like a different response so you naively modify the system parameters:
            \begin{align*}
        m_c = \SI{1}{\kilo\gram} \quad m_{pm} = \SI{0.01}{\kilo\gram} \quad b = \SI{0.02}{\newton\second\per\meter} \quad k = \SI{0.1}{\newton\per\meter}.
            \end{align*}
            What is the output response if \( f(t) = \SI{30}{\newton} \) and zero initial conditions. 
        \item Generate a plot with all three output responses.
    \end{subprob}
\end{prob}
   
\begin{prob}
    You're trying to ``tune'' a high pass filter,  by simply guessing no less!
    The filter is defined by
    \begin{align}
    \bracket{s + \frac{R_1 + R_2}{R_1 R_2 C} } V_{out}(s) = s V_{in}(s) + V_{out}(0) ,
    \end{align}
    where \( R_1 = \SI{40}{\kilo\ohm}, R_2 = \SI{200}{\kilo\ohm}, C = \SI{5}{\micro\farad}\).
    \begin{subprob}
    \item Find the output response for \( v_{in}(t) = 10 \sin 0.6 t \) with zero initial conditions.
    \item Find the output response for \( v_{in}(t) = 10 \sin 60 t \) with zero initial conditions.
    \item  Generate a plot with both responses and discuss how effective this design is for a \textbf{HIGH} pass filter.
    \end{subprob}
\end{prob}

\begin{prob}
    A Boeing 747 is in Straight and Level Unaccelerated Flight (SLUF) at \( \SI{40000}{ft}, \SI{871}{ft \per \second}\).
    The linearized equations of longitudanal motion are given by
    \begin{align}
        \begin{bmatrix}
            \begin{array}{ccc}
                s + 0.0828 & -0.0215 & 0.5589 \\
                0.0573 & 15.3348s + 5.9633 & -15.0685s \\
                0.0057 & 0.1425s + 1.6165 & s^2 + 0.0438s
            \end{array}
        \end{bmatrix}
        \begin{bmatrix}
            \begin{array}{c}
                \delta u \\ \delta \alpha \\ \delta \theta
            \end{array}
        \end{bmatrix}
        =
        \begin{bmatrix}
            \begin{array}{c}
                0 \\ -0.3226 \\ -1.2124
            \end{array}
        \end{bmatrix}
        \delta e ,
    \end{align}
    where \( \delta u, \delta \theta, \delta e, \delta \alpha\) are the deviations of forward velocity in \si{ft \per \second}, pitch in degrees, elevation deflection in degree, and deviation of the angle of attack in degrees, respectively.

We can define a transfer function relating the input elevator deflection to the output of the pitch devaitions as
\begin{align}
    \frac{\theta(s)}{\delta e(s)} = \frac{-18.5 (s + 0.36) (s+0.08)}{(s + 0.47 \pm 1.24 j)(s+0.06)(s+0.02)} .
\end{align}

\begin{subprob}
    \item Find the pitch response \( \theta (t) \) if the elevation is deflected a constant \SI{-0.01}{\degree}.
\end{subprob}
\end{prob}

It is typically difficult to analyze the short period motion, due to the complex poles, and the phugoid motion, due to the real poles. 
As a result, it is possible to seperate the dynamics into two seperate transfer functions:
\begin{align}
    \frac{\delta \alpha}{\delta e} &= \frac{-0.021 (s^2 + 56.7 s)}{s(s+0.283 \pm 1.24 j)} \\
    \frac{\delta \theta}{\delta e} &= \frac{-1.21 (s + 0.36)}{s(s+0.283 \pm 1.24 j)}
\end{align}

\begin{prob}
    Insert a tikz picture of a missle in flight
    A missle in flight, as shown in Fig whatever, is subject to several forces, such as thrust, lift, drag, and gravity. 
    The missle flies at an angle of attack, \( \alpha \), with respect to the velocity vector, creating lift.
    For steering, the body angle from vertical , \( \phi\), is controlled by rotating the thrust vector at the tail.
    The transfer function relating the body angle, \( \phi\), to the angular displacement, \( \delta \), of the engine is of the form
    \begin{align}
        \frac{\phi(s)}{\delta (s)} = \frac{K_a s + K_b}{K_3 s^3 + K_s s^2 + K_1 s + K_0}
    \end{align}
\end{prob}

\begin{prob}
    Add a picture

    The transfer function relating the pitch angle \( \theta (s) \) to the elevator deflection angle \( \delta e (s) \) for an Unmanned Free-Swimming Submersible is given by
    \begin{align}
        \frac{\theta (s) }{\delta e(s)} = \frac{-0.125 (s + 0.435)}{(s+1.23)(s^2 + 0.226 s + 0.0169)} .
    \end{align}

    \begin{subprob}
    \item Using only the second order poles from the transfer function predict the percent overshoot, rise time, peak time and settling time.
    \item Using the Laplace transform, find the analytical expression for the pitch angle response to a step input of the elevator.
    \item Evaluate the effect of teh additional pole and zero on the validity of the second order approximation.
    \item Plot the step response of the vheicle dynamics and verify your conclusions found in the previous problem.
    \end{subprob}
\end{prob}

\begin{prob}
    Ships at sea undergo motion about their roll axis.
    Fins called stabiliers are used to reduce this rolling motion.
    The stabilizers can be positioned by a closed-loop roll control system that consists of components, such as fin actuators and sensors, as well as the ships roll dynamics.
    Assume the roll dynamics, which relates the roll angle output, \( \theta (s) \) to a disturbance torque input, \( T_D(s)\) is given by
    \begin{align}
        \frac{\theta(s)}{\T_D(s)} = \frac{2.25}{s^2 + 0.5 s + 2.25} .
    \end{align}

    Accomplish the following:
    \begin{subprob}
        \item Find the natural frequency, damping ratio, peak time, settling time, rise time, and percent overshoot.
        \item Find the analytical expression for the output response to a unit step input.
        \item Use a computer tool to verify your solutions and plot the response to a step input.
    \end{subprob}
\end{prob}
\end{document}

