
\documentclass[10pt]{article}
% \usepackage[letterpaper,text={6.5in,8.7in},centering]{geometry}
\usepackage{amssymb,amsmath,times,url,graphicx,amsthm,alltt}
%\usepackage[pdftex,urlcolor=blue,pdfpagemode=none,pdfstartview=FitH]{hyperref}
\usepackage{my_packages}
\usepackage{tikz_packages}
%% url smaller font.
\makeatletter
\def\url@leostyle{%
\@ifundefined{selectfont}{\def\UrlFont{\sf}}{\def\UrlFont{\small\ttfamily}}}
\makeatother
\urlstyle{leo}

%\usepackage[all,import]{xy}

\renewcommand{\baselinestretch}{1.2}
\date{}

\renewcommand{\thesubsection}{\arabic{subsection}. }
\renewcommand{\thesubsubsection}{\arabic{subsection}.\arabic{subsubsection} }

\theoremstyle{definition}
\newtheorem{prob}{Problem}[section]
%\renewcommand{\theprob}{\arabic{section}.\arabic{prob}}
\renewcommand{\theprob}{\arabic{prob}}

\newenvironment{subprob}%
{\renewcommand{\theenumi}{\alph{enumi}}\renewcommand{\labelenumi}{(\theenumi)}\begin{enumerate}}%
{\end{enumerate}}%


\begin{document}

\pagestyle{empty}
\section*{MAE3134: Homework 5}
\vspace*{-0.4cm}
\noindent{Due date: }%\\%\vspace*{0.5cm}

\begin{prob}
    For each of the electrical systems below, find the state space representation.
    \begin{subprob}
    \item There is a single voltage source and the output is the voltage difference across the capacitor.
        \begin{figure}[h]
            \centering
            \begin{scaletikzpicturetowidth}{0.8\textwidth}
                \begin{tikzpicture}[scale=\tikzscale]
                    \draw (0,0) to[american voltage source,v<=\( u(t)\)] (0,2) 
                        to[resistor, l^=$R_1$] (2,2) to[inductor, l=$H_1$] (2,0) 
                        to[short] (0,0); 
                    \draw (2, 2) to[resistor, l^=$R_2$] (4, 2) to[inductor, l=$H_2$] (4, 0) to[short] (2, 0);
                    \draw (4, 2) to[resistor, l=$R_3$] (6, 2) to[capacitor, l=$C_1$] (6, 0) to[short] (4, 0);
                \end{tikzpicture}
            \end{scaletikzpicturetowidth}

            \caption{Electrical System}
        \end{figure}

    \item The output is \( i(t) \), which defines the current across the resistor \( R_2\). 
        \begin{figure}[h]
            \centering
            \begin{scaletikzpicturetowidth}{0.8\textwidth}
                \begin{tikzpicture}[scale=\tikzscale]
                    \draw (0,0) to[american voltage source,v<=\( u_1(t)\)] (0,2) 
                        to[inductor, l^=$R_1$] (2,2) to[capacitor, l=$H_1$] (2,0) 
                        to[short] (0,0); 
                    \draw (2, 2) to[capacitor, l^=$H_1$] (4, 2) to[american voltage source, l=$u_2(t)$] (4, 0) to[short] (2, 0);
                    \draw (4, 2) to[short] (6, 2) to[resistor, l=$R_2$] (6, 0) to[short] (4, 0);
                \end{tikzpicture}
            \end{scaletikzpicturetowidth}
        \end{figure}
    \item The output is \( v_o(t) \) defining the voltage across the resistor \( R_2\).
        \begin{figure}[h]
            \centering
            \begin{scaletikzpicturetowidth}{0.5\textwidth}
                \begin{tikzpicture}[scale=\tikzscale]
                    \draw (0,0) to[american voltage source,v<=\( u_1(t)\)] (0,2) 
                        to[resistor, l^=$R_1$] (2,2) to[capacitor, l=$H_1$] (2,0) 
                        to[short] (0,0); 
                    \draw (2, 2) to[capacitor, l^=$H_2$] (4, 2) to[resistor, l=$R_3$] (4, 0) to[short] (2, 0);
                    \draw (0, 2) to[short] (0, 4) to[resistor, l=$R_2$] (4, 4) to[short] (4, 2);
                \end{tikzpicture}
            \end{scaletikzpicturetowidth}
            \caption{Electrical System}
        \end{figure}
    \end{subprob}
\end{prob}

\end{document}
