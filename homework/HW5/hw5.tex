\documentclass[10pt]{article}
\usepackage{my_packages}
\usepackage{tikz_packages}
%% url smaller font.
\makeatletter
\def\url@leostyle{%
\@ifundefined{selectfont}{\def\UrlFont{\sf}}{\def\UrlFont{\small\ttfamily}}}
\makeatother
\urlstyle{leo}

%\usepackage[all,import]{xy}

\renewcommand{\baselinestretch}{1.2}
\date{}

\renewcommand{\thesubsection}{\arabic{subsection}. }
\renewcommand{\thesubsubsection}{\arabic{subsection}.\arabic{subsubsection} }

\theoremstyle{definition}
\newtheorem{prob}{Problem}[section]
%\renewcommand{\theprob}{\arabic{section}.\arabic{prob}}
\renewcommand{\theprob}{\arabic{prob}}

\newenvironment{subprob}%
{\renewcommand{\theenumi}{\alph{enumi}}\renewcommand{\labelenumi}{(\theenumi)}\begin{enumerate}}%
{\end{enumerate}}%


\begin{document}

\pagestyle{empty}
\section*{MAE3134: Homework 5}
\vspace*{-0.4cm}
\noindent{Due date: 29 March 2018 }%\\%\vspace*{0.5cm}

\begin{prob}
    For each of the electrical systems below, find the state space representation.
    \begin{subprob}
    \item There is a single voltage source and the output is the voltage difference across the capacitor \( C_1 \).
        \begin{figure}[h]
            \centering
            \begin{scaletikzpicturetowidth}{0.8\textwidth}
                \begin{tikzpicture}[scale=\tikzscale]
                    \draw (0,0) to[american voltage source,v<=\( u(t)\)] (0,2) 
                        to[resistor, l^=$R_1$] (2,2) to[inductor, l=$H_1$] (2,0) 
                        to[short] (0,0); 
                    \draw (2, 2) to[resistor, l^=$R_2$] (4, 2) to[inductor, l=$H_2$] (4, 0) to[short] (2, 0);
                    \draw (4, 2) to[resistor, l=$R_3$] (6, 2) to[capacitor, l=$C_1$] (6, 0) to[short] (4, 0);
                \end{tikzpicture}
            \end{scaletikzpicturetowidth}

            \caption{Electrical System}
        \end{figure}

    \item The output is \( i(t) \), which defines the current across the resistor \( R_2\). 
        \begin{figure}[h]
            \centering
            \begin{scaletikzpicturetowidth}{0.8\textwidth}
                \begin{tikzpicture}[scale=\tikzscale]
                    \draw (0,0) to[american voltage source,v<=\( u_1(t)\)] (0,2) 
                        to[inductor, l^=$H_1$] (2,2) to[capacitor, l=$C_2$] (2,0) 
                        to[short] (0,0); 
                    \draw (2, 2) to[capacitor, l^=$C_1$] (4, 2) to[american voltage source, l=$u_2(t)$] (4, 0) to[short] (2, 0);
                    \draw (4, 2) to[short] (6, 2) to[resistor, l=$R_2$] (6, 0) to[short] (4, 0);
                \end{tikzpicture}
            \end{scaletikzpicturetowidth}
        \end{figure}
    \item The output is \( v_o(t) \) which defines the voltage across the resistor \( R_3\).
        \begin{figure}[h]
            \centering
            \begin{scaletikzpicturetowidth}{0.5\textwidth}
                \begin{tikzpicture}[scale=\tikzscale]
                    \draw (0,0) to[american voltage source,v<=\( u_1(t)\)] (0,2) 
                        to[resistor, l^=$R_1$] (2,2) to[capacitor, l=$C_1$] (2,0) 
                        to[short] (0,0); 
                    \draw (2, 2) to[capacitor, l^=$C_2$] (4, 2) to[resistor, l=$R_3$] (4, 0) to[short] (2, 0);
                    \draw (0, 2) to[short] (0, 4) to[resistor, l=$R_2$] (4, 4) to[short] (4, 2);
                \end{tikzpicture}
            \end{scaletikzpicturetowidth}
            \caption{Electrical System}
        \end{figure}
    \end{subprob}
\end{prob}

\begin{prob}
    \begin{figure}[h]
        \centering
        \begin{scaletikzpicturetowidth}{1\textwidth}
            \begin{tikzpicture}[scale=\tikzscale]
            \draw (0, 0) node[inner sep=0] {\includegraphics[width=0.7\textwidth]{figures/ArWeg.png}};
            \draw (-2, 1.4) node {\(\alpha\)};
        \end{tikzpicture}
    \end{scaletikzpicturetowidth}
    \caption{Missile in flight~\label{fig:missile}}
    \end{figure}
    A missile in flight, as shown in~\cref{fig:missile}, is subject to several forces: thrust, lift, draft, and gravity.
    The missile flies at an angle of attack \( \alpha \), from its longitutinaal axis, creating lift. 
    For steering, the body angle from vertical, \( \phi \), is controlled by rotating the engine at the tail. 
    The transfer function relating the body angle, \( \phi \), to the angular displacement \( \delta \) of the engine is of the form
    \begin{align*}
        \frac{\Phi(s)}{\delta(s)} = \frac{K_a s + K_b}{K_3 s^3 + K_2 s^2 + K_1 s + K_0}.  
    \end{align*}

    \noindent Find the representation of the missile steering control in state space.
\end{prob}

\clearpage\newpage
\begin{prob}
     \begin{figure}[h]
        \centering
        \subcaptionbox{F-4E with canards\label{fig:f4}}{\includegraphics[width=0.5\textwidth]{figures/f4e.png}}~
        \subcaptionbox{F-4E in flight\label{fig:qf4e}}{\includegraphics[width=0.5\textwidth]{figures/QF-4_Holloman_AFB.jpg}}
     \end{figure}
     The McDonnell Douglas F-4 Phantom II is a tandem two-seat, twin-engine, all-weather, long-range supersonic jet interceptor and fighter-bomber.
     First entering service in 1960, it proved highly adaptable and was a major part of the air wings of three service components, the US Navy, US Marine Corps and US Air Force.
     The F-4 was used extensively  during the Vietnam War and served as the principal air superiority fighter for both the Navy and Air Force.
     The F-4 remained in active use through the 1991 Gulf War serving in reconnaissance and Wild Weasel (Suppression of Enemy Air Defenses) roles.
    
     Normal accelerations, \( a_n \), and pitch rate, \( q \), are controlled by elevator deflection, \( \delta_e\), on the horizontal stabilizers and by canard deflection, \( \delta_c \).
    A commanded deflection , \( \delta_{com} \), is used to effect a change in both \( \delta_e \) and \( \delta_c\).
    The actuator deflections, combined with the aircraft longitudinal dynamics yield \( a_n \) and \( q\).
    The state equations describing the effect of \( \delta_{com} \) on \( a_n \) and \( q\) is given by
    \begin{align*}
        \begin{bmatrix} 
            \dot a_n \\ \dot q \\ \dot \delta_e 
        \end{bmatrix}
        = \begin{bmatrix}
            -1.702 & 50.72 & 263.38 \\
            0.22 & -1.418 & -31.99\\
            0 & 0 & -14
        \end{bmatrix}
        \begin{bmatrix}
            a_n \\ q \\ \delta_e
        \end{bmatrix}
        + 
        \begin{bmatrix}
            -272.06 \\ 0 \\14
        \end{bmatrix}
        \delta_{com}
    \end{align*} .

    \noindent Find the following transfer functions:
    \begin{subprob}
        \item
            \[
                G_1(s) = \frac{A_n(s)}{\delta_{com}(s)}
            \]
        \item 
            \[
                G_2(s) = \frac{Q(s)}{\delta_{com}(s)}
            \]
    \end{subprob}
\end{prob}

\clearpage\newpage
\begin{prob}
    An autopilot is to be designed for a submarine as shown in~\cref{fig:submarine} to maintain a constant depth under severe wave disturbances.
    \begin{figure}[h]
        \centering
        \includegraphics[width=0.4\textwidth]{figures/submarine.png}
        \caption{Submarine pitch axis control~\label{fig:submarine}}
    \end{figure}
    This system has two inputs and two outputs in contrast to classical control methods of single input and single output systems.
    The linearized dynamics under neutral buoyancy and at a constant speed are given by
    \begin{align*}
        \dot{\vecbf{x}} &= A \vecbf{x} + B \vecbf{u} \\
        \vecbf{y} &= C \vecbf{x}
    \end{align*}
    where
    \begin{align*}
        A &= \begin{bmatrix}
            -0.038 & 0.896 & 0 & 0.0015 \\
            0.0017 & -0.092 & 0 & -0.0056 \\
            1 & 0 & 0 & -3.086 \\
            0 & 1 & 0 & 0
        \end{bmatrix}
        \quad 
        B = \begin{bmatrix}
            -0.0075 & -0.023 \\
            0.0017 & -0.0022 \\
            0 & 0 \\
            0 & 0
        \end{bmatrix} \\
        C &= \begin{bmatrix}
            0 & 0 & 1 & 0 \\
            0 & 0 & 0 & 1
        \end{bmatrix}
    \end{align*}. 

    The state, output and input are defined as
    \begin{align*}
        \vecbf{x} = \begin{bmatrix}
            w \\ q \\ z \\ \theta
        \end{bmatrix}
        \quad
        \vecbf{y} = \begin{bmatrix} z \\ \theta \end{bmatrix} \quad \vecbf{u} = \begin{bmatrix} \delta_B \\ \delta_S \end{bmatrix}
    \end{align*}
    where \( w\) is the heave velocity, \( q \) is the pitch rate, \( z \) is the submarine depth, \( \theta\) is the pitch angle, \( \delta_B\) is the bow hydroplane angle, and \( \delta_S\) is the stern hydroplane angle.

    \begin{subprob}
        \item Use Matlab/Python to find the transfer function matrix. 
            Recall there should be four possible transfer functions.
        \item Using the previous results write the transfer functions for the following input/output combinations.
            \begin{align*}
                \frac{z(s)}{\delta_B(s)} ,\quad \frac{z(s)}{\delta_S(s)} ,\quad \frac{\theta(s)}{\delta_B(s)} ,\quad \frac{\theta(s)}{\delta_S(s)}
            \end{align*}
    \end{subprob}
\end{prob}

\begin{prob}
    Linearize (if possible) the following systems about \textbf{EACH} of the their equilibrium states, if not possible state why, and obtain the state matrix, \( A \), for these linearized systems.

    \begin{subprob}
        \item \[ \dot x = x^3\]
        \item \[ \dot x = \sqrt{ \abs{x}} \]
        \item The following scalar equations are one form of Euler's equations for the rotational motion of a rigid body. 
            You may assume that the body is unsymmetric, i.e. \( I_1 \neq I_2 \neq I_3\).
            \begin{align*}
                I_1 \dot \omega_1 &= (I_2 - I_3) \omega_2 \omega_3 , \\
                I_2 \dot \omega_2 &= (I_3 - I_1) \omega_3 \omega_1 , \\
                I_3 \dot \omega_3 &= (I_1 - I_2) \omega_1 \omega_2  \\
            \end{align*}
    \end{subprob}
\end{prob}

\begin{prob}
    Linearize about each equilibrium point and find the state matrix, \( A \) for the state space representation of the linearized system.

    \begin{subprob}
        \item \[ \ddot y + \parenth{y^2 - 1} \dot y + y = 0\] where \( y(t) \) is a scalar.
        \item \[ \ddot y + \dot y + y - y^3 = 0 \] where \( y(t) \) is a scalar.
        \item \begin{align*}
                \parenth{M + m} \ddot y + ml \ddot \theta \cos \theta - ml \dot \theta^2 \sin \theta + ky &= 0, \\
                m l \ddot y \cos \theta + ml^2 \ddot \theta + m g l \sin \theta &=0.
            \end{align*}
    
    \end{subprob}
\end{prob}

\begin{prob}
    Obtain the transfer function (matrix) for the following system
    \begin{align*}
        \ddot y_1 + \ddot y_2 + y_1 + y_2 &=  u_1 + \dot u_2 , \\
        2 \ddot y_1 + 3 \ddot y_2 + y_1 - y_2 &= 0 .
    \end{align*}
\end{prob}

\begin{prob}
    Obtain the transfer function of the system with input \( u \) and output \( y \) described by
    \begin{align*}
        \ddot q_1 + 3 \dot q_2 + \dot q_1 + 2 q_2 &= \dot u + 4 u , \\
        \ddot q_1 + 4 \dot q_2 + 3 q_2 &= u , \\
        y & =  q_1 + q_2.
    \end{align*}
\end{prob}

\begin{prob}
    Obtain the transfer funciton for the following system
    \begin{align*}
        \dot x(t) &= - x(t) + 2 x(t-h) + u(t) , \\
        y(t) &= x(t)
    \end{align*}
\end{prob}
\end{document}
