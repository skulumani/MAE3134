\documentclass[11pt, reqno]{article}   	% use "amsart" instead of "article" for AMSLaTeX format
\usepackage{my_packages}
\usepackage{parskip}
\usepackage{biblatex}
\addbibresource{library.bib}

\title{MAE 3134: Linear System Dynamics}
\author{Shankar Kulumani}
\date{Spring 2017}							% Activate to display a given date or no date

\begin{document}
{\noindent\Large \textbf{MAE 3134: Linear System Dynamics}}

Spring 2017\\
\textbf{Lecture}: Tuesday and Thursday 0935-1050, 1957 E 214  \\
\textbf{Recitation}: Friday 1420-1510, DUQUES 359
\paragraph{Instructor}
\begin{minipage}[t]{0.8\textwidth}
Shankar Kulumani \quad Email:\href{mailto:skulumani@gwu.edu}{skulumani@gwu.edu}\\
Office Hours: SEH 2200, M 1400-1600, and by appointment
\end{minipage}

\paragraph{Teaching Assistant}
\begin{minipage}[t]{0.8\textwidth}
Evan Kaufman \quad Email:\href{mailto:evankaufman@gwu.edu}{evankaufman@gwu.edu}\\
\end{minipage}

\paragraph{Prerequisites}
\begin{minipage}[t]{0.8\textwidth}
ASPC 2113 and ASPC 2058\\
Ordinary Differential Equations, Complex Numbers, Linear Algebra, Scientific Programming (\textsc{Matlab}, Python, etc.)
\end{minipage}

\paragraph{Course Goal} 
This course will introduce students to the \textit{fundamentals of linear systems analysis}.
Students will learn techniques to describe simple electro-mechanical system in terms of a \textit{mathematical model}.
Using this model, students will learn how to predict the behavior of the system through both analytical and numerical techniques.

\paragraph{Textbook}
\fullcite{ogata2004}

\paragraph{Additional Resources} 
Some additional references for your use.
\begin{itemize}
    \item \fullcite{nise2004}
    \item \fullcite{phillips1995}
    \item \fullcite{goodwin2001}
\end{itemize}

\paragraph{Course Objectives}
\begin{enumerate}
    \item Determine the equations of motion for simple mechanical systems.
    \item Derive and solve a differential equation of motion for simple mechanical and electrical systems.
    \item Describe and predict the performance of first and second order linear systems using time and frequency domain techniques.
    \item Produce the Bode frequency response plot for a linear system.
    \item Design parameter changes for open and closed loop linear systems in order to meet system performance specifications using s-plane tools and Bode plots.
\end{enumerate}

\paragraph{Grading}
\begin{tabular}[t]{lr}
    Homework & \SI{40}{\percent} \\
    Recitation & \SI{10}{\percent}\\
    Midterm exam & \SI{25}{\percent}\\
    Final exam & \SI{25}{\percent}
\end{tabular}

\paragraph{Topics \& Schedule}
\begin{table}[h]
\centering
\begin{tabular}{lr}
    Week & Topic \\ \hline
    1 & Course Introduction \\
    2 & System Modelling \\
    3 & Laplace Transform-Introduction\\ 
    4 & Laplace Transform-Solving ODEs\\
    5 & Transfer Function\\
    6 & Transfer Function\\
    7 & State Space \\
    8 & Review for Midterm\\
    9 & System Response - First Order\\
    10 & System Response - Second Order\\
    11 & System Behavior -Stability\\
    12 & Frequency Response\\
    13 & Bode Plot\\
    14 & Extra\\
    15 & Review for Final\\
\end{tabular}
\caption{Course Schedule}
\end{table}

\begin{itemize}
    \item System Modeling
        \begin{enumerate}
            \item System Modeling
            \item Laplace Transform
            \item State space models
        \end{enumerate}
  
    \item System Response
        \begin{enumerate}
            \item Transfer and Response Functions
            \item Solution method for time response
            \item Performance criteria for first and second order systems
        \end{enumerate}
    \item Frequency Domain
        \begin{enumerate}
            \item Fourier analysis
            \item Bode Plots
        \end{enumerate}
\end{itemize}

\paragraph{Attendance Policy} 
Students are expected to attend every class session.
All absences require prior instructor approval.
This means you should personally contact your instructor \textbf{prior} to missing a class.
Last minute e-mail messages are \textbf{unacceptable}.

\paragraph{Homework Policy}

\begin{itemize}
    \item Approximately one assignment per week
    \item All graded work is due at the \textbf{beginning} of class.
    \item \textbf{Late homework} will \textbf{NOT} be accepted for any reason.
    \item Homework grading will be based on your ability to present the solution in a clear and neat fashion.
    An engineer's work must be understood by others, with each step of their work understandable and reproducible. 
    The ability to write clear, professional, well-organized documents and reports is an essential skill you should hone now--it is critical to any profession.
    \begin{itemize}
        \item Use one side of a clean sheet of letter paper (graphed, lined, blank are all acceptable).
        \textbf{Note:} Not from a spiral notebook
        \item Write your name clearly on the first page
        \item Clearly number your solutions and final answer using a box or some other method.
        \item Assignments should be written/typed clearly and legibly. 
        Any unacceptable work will be returned.
        \item Use a stapler.
    \end{itemize}
    \item All homework is \textbf{individual effort}.
    Students may discuss homework problems with others to develop and clarify their approach.
    However, the written solution, or computer programming, should be an independent and individual effort that reflects the personal understanding of the material.
    \textbf{Any copying or integrity violation will not be tolerated.}
\end{itemize}

\paragraph{Exam Policy}
There is one midterm exam and one final exam. 
Make-up exams will only be given in \textbf{exceptional circumstances}.
Students should notify the instructor \textbf{as soon as possible} in the case of any scheduling conflicts.

\paragraph{University Policy on Religious Holidays}
\begin{enumerate}
    \item Students should notify faculty during the first week of the semester of their intention to be absent from class on their day(s) of religious observance.
    \item Faculty should extend to these students the courtesy of absence without penalty on such occasions, including permission to make up examinations.
    \item Faculty who intend to observe a religious holiday should arrange at the beginning of the semester to reschedule missed classes or to make other provisions for their course-related activities.
\end{enumerate}

\paragraph{Support for Students Outside the Classroom}
\begin{enumerate}
    \item Disability Support Services (DSS)
    Any student who may need an accommodation based on the potential impact of a disability should contact the Disability Support Services office at 202-994-8250 in the Rome Hall, Suite 102, to establish eligibility and to coordinate reasonable accommodations. 
    For additional information please refer to: \href{https://disabilitysupport.gwu.edu/}{\texttt{https://disabilitysupport.gwu.edu/}}
    \item Mental Health Services 202-994-5300
    The University's Mental Health Services offers 24/7 assistance and referral to address students' personal, social, career, and study skills problems. 
    Services for students include: crisis and emergency mental health consultations confidential assessment, counseling services (individual and small group), and referrals.
    \href{https://counselingcenter.gwu.edu/}{\texttt{https://counselingcenter.gwu.edu/}}
\end{enumerate}

\paragraph{Academic Integrity}
Academic dishonesty is defined as cheating of any kind, including misrepresenting one's own work, taking credit for the work of others without crediting them and without appropriate authorization, and the fabrication of information. 
For the remainder of the code, see: \href{https://studentconduct.gwu.edu/code-academic-integrity}{\texttt{https://studentconduct.gwu.edu/code-academic-integrity}}
% \printbibliography
\end{document}  